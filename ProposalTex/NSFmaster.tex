\documentclass{article}

%%%%%%%%%%%%%%%%%%%%%%%%%%%%%%%%%%%%%%%%%%%%%%%%%%%%%%%%%%%%%%%%%%%%%%%%%
\pagestyle{plain}                                                      %%
%%%%%%%%%% EXACT 1in MARGINS %%%%%%%                                   %%
\setlength{\textwidth}{6.5in}     %%                                   %%
\setlength{\oddsidemargin}{0in}   %% (It is recommended that you       %%
\setlength{\evensidemargin}{0in}  %%  not change these parameters,     %%
\setlength{\textheight}{8.5in}    %%  at the risk of having your       %%
\setlength{\topmargin}{0in}       %%  proposal dismissed on the basis  %%
\setlength{\headheight}{0in}      %%  of incorrect formatting!!!)      %%
\setlength{\headsep}{0in}         %%                                   %%
\setlength{\footskip}{.5in}       %%                                   %%
%%%%%%%%%%%%%%%%%%%%%%%%%%%%%%%%%%%%                                   %%
\newcommand{\required}[1]{\section*{\hfil #1\hfil}}                    %%
\renewcommand{\refname}{\hfil References Cited\hfil}                   %%
\bibliographystyle{plain}                                              %%
%%%%%%%%%%%%%%%%%%%%%%%%%%%%%%%%%%%%%%%%%%%%%%%%%%%%%%%%%%%%%%%%%%%%%%%%%

%PUT YOUR MACROS HERE

%
%%%%%%%%% SUMMARY -- 1 page, third person
% e.g:  "The PI will prove" not "I will prove"

\required{Project Summary}
\required{Overview}
The CORAL project consists of a hierarchical multilayer data processing system, that contains the following Levels (with reference to EOSIS [https://science.nasa.gov/earth-science/earth-science-data/data-processing-levels-for-eosdis-data-products])
[Insert table here containing Table 1. https://coral.jpl.nasa.gov/data-products]

\documentclass{article}
\usepackage[utf8]{inputenc}
\usepackage[table]{xcolor}
 
\setlength{\arrayrulewidth}{1mm}
\setlength{\tabcolsep}{18pt}
\renewcommand{\arraystretch}{2.5}
 
\newcolumntype{s}{>{\columncolor[HTML]{AAACED}} p{3cm}}
 
\arrayrulecolor[HTML]{DB5800}
 
\begin{tabular}{ |s|p{3cm}|p{3cm}|  }
\hline
\rowcolor{lightgray} \multicolumn{3}{|c|}{Country List} \\
\hline
Country Name    or Area Name& ISO ALPHA 2 Code &ISO ALPHA 3 \\
\hline
Afghanistan & AF &AFG \\
\rowcolor{gray}
Aland Islands & AX & ALA \\
Albania   &AL & ALB \\
Algeria  &DZ & DZA \\
American Samoa & AS & ASM \\
Andorra & AD & \cellcolor[HTML]{AA0044} AND    \\
Angola & AO & AGO \\
\hline
\end{tabular}

Level 0 processing operates on the raw data generated from the spectroscopic instrument. As is commonplace with legacy hardware, sensitivity measurements are recorded in a range of different file formats that contains instrument specific details. A significant amount of time must be invested by the researcher to process these configuration files in order to generate data that can then be used in the context of the research. A system must be created that can efficiently extract data across various configuration formats, and consolidate the information to create datasets containing a higher degree of coherency. 

\required{Intellectual Merit}
% This should describe the potential of the proposed activity
% to advance knowledge in the field of mathematics. 

\required{Broader Impacts}
% This should describe the potential of the proposed activity
% to benefit society and contribute to the achievement of 
% specific, desired societal outcomes.
% Examples of Broader Imacts include, but are not limited to:
% 1. Full participation of women, persons with disabilities, and 
% underrepresented minorities in science, technology, engineering, 
% and mathematics (STEM); 
% 2. Improved STEM education and educator 
% development at any level; 
% 3. Increased public scientific literacy 
% and public engagement with science and technology; 
% 4. Improved well-being of individuals in society; 
% 5. Development of a diverse, globally competitive STEM workforce;
% 6. Increased partnerships between academia, industry, and others; 
% 7. Improved national security; 
% 8. Increased economic competitiveness of the US; 
% 9. Enhanced infrastructure for research and education.


% \setcounter{page}{1}
%\include{NSFdesc}
% \setcounter{page}{1}
% \include{NSFrefs}
% \setcounter{page}{1}
% 
%%%%%%%%% BIOGRAPHICAL SKETCH -- 2 pages

\required{Biographical Sketch: Your Name}

% Your Bio should be divided into the following sections
% It is not required to include the parenthecital letters preceding
\required{(a) Professional Preparation}
% This is your educational background:
% Undergrad, Location, Major, Year
% Graduate, Location, Major, Year
% Postdoc, Location, Area, Years-Inclusive
\required{(b) Appointments}
% List most recent first
\required{(c) Products}
% List up to 5 related to the proposal, and up to 5 "Other Significant Products"
% Must be citable and accessible.
% Including, but not limited to: publications, data sets, software, patents and copyrights.
% Unacceptable product examples are: unpublished documents
% (not yet submitted for publication) and invited lectures.
% Each product must include full citation information.
% Including (as applicable and available) names of authors,
% date of publication or release, title, title of enclosing work such
% as journal or book, volume, issue, pages, website and URL.
% If only publications are included, you may use the header "Publications"
\required{(d) Synergistic Activities}
% List up to 5 eaxamples
% Per NSF guidelines, they should "demonstrate the broader impact 
% of the individual's professional and scholarly activities 
% that focuses on the integration and transfer of knowledge
% along with its creation".
% Examples include, but not limited to: innovations in teaching and training
% (i.e. development of curricular materials and pedagogical methods);
% contributions to the science of learning; development and/or refinement 
% of research tools; computation methodologies, and algorithms for 
% problem-solving; development of databases to support research 
% and education; broadening the participation of groups underrepresented in STEM; 
% and service to the scientific and engineering community outside 
% of the individual’s immediate organization.)




% \setcounter{page}{1}
%%%%%%%%%% DATA MANAGEMENT PLAN -- 2 pages
\required{Data Management Plan}
% Include this supplementary document for your plans for data management
% and sharing of the products of research.
% Describe how this proposal will conform to NSF policy on the 
% dissemination and sharing of research results.
% This may incude
% 1. the types of data, samples, physical collections, software, 
% curriculum materials, and other materials to be produced in the course of the project;
% 2. the standards to be used for data and metadata format 
% and content (where existing standards are absent or deemed inadequate, 
% this should be documented along with any proposed solutions or remedies);
% 3. policies for access and sharing including provisions for appropriate 
% protection of privacy, confidentiality, security, intellectual property, 
% or other rights or requirements;
% 4. policies and provisions for re-use, re-distribution, 
% and the production of derivatives; and
% 5. plans for archiving data, samples, and other research products, 
% and for preservation of access to them.
% A valid Data Management Plan may include only the statement 
% that no detailed plan is needed, as long as the statement is 
% accompanied by a clear justification.
% \setcounter{page}{1}
%\include{NSFcollab}
% \setcounter{page}{1}

\usepackage{amsmath}
\usepackage{amssymb}
\usepackage[landscape]{geometry}
\usepackage{tikz}
\usetikzlibrary{timeline}
\usepackage{booktabs}
\usepackage[table,xcdraw]{xcolor}
^{•}

\pagestyle{empty}
\begin{document}

%%%%%%%%% SUMMARY -- 1 page, third person
% e.g:  "The PI will prove" not "I will prove"

\required{Project Summary}
\section{Overview}
The CORAL project consists of a hierarchical multilayer data processing system, that contains the following Levels (with reference to EOSIS [https://science.nasa.gov/earth-science/earth-science-data/data-processing-levels-for-eosdis-data-products])
[Insert table here containing Table 1. https://coral.jpl.nasa.gov/data-products]

\begin{table}[]
\centering
\caption{My caption}
\label{my-label}
\begin{tabular}{@{}ll@{}}
\toprule
\rowcolor[HTML]{656565} 
{\color[HTML]{FFFFFF} Data Product}                         & {\color[HTML]{FFFFFF} Description}                                                                                                                                                                      \\ \midrule
Level 0                                                     & \begin{tabular}[c]{@{}l@{}}Reconstructed, unprocessed PRISM digitized numbers (DN)\\ at full resolution with GPS\end{tabular}                                                                           \\
Level 1                                                     & \begin{tabular}[c]{@{}l@{}}Calibrated spectral radiance with geolocation information including\\ illumination and observation geometry\end{tabular}                                                     \\
Level 2                                                     & \begin{tabular}[c]{@{}l@{}}Benthic reflectance generated following atmosphere and water column\\ radiative transfer inversion with geolocation, support processing\\ information and flags\end{tabular} \\ \midrule
Level 3                                                     & \begin{tabular}[c]{@{}l@{}}Benthic cover, i.e., seafloor classified into coverage of benthic types\\ (coral, algae, sand) with geolocation, uncertainties, and flags\end{tabular}                       \\
Level 4                                                     & Benthic primary productivity and calcification                                                                                                                                                          \\
\begin{tabular}[c]{@{}l@{}}Ancillary in\\ situ\end{tabular} & \begin{tabular}[c]{@{}l@{}}Optical, benthic cover, and benthic community productivity\\ and calcification calibration/validation\end{tabular}                                                          
\end{tabular}
\end{table}


Level 0 processing operates on the raw data generated from the spectroscopic instrument. As is commonplace with legacy hardware, sensitivity measurements are recorded in a range of different file formats that contains instrument specific details. A significant amount of time must be invested by the researcher to process these configuration files in order to generate data that can then be used in the context of the research. A system must be created that can efficiently extract data across various configuration formats, and consolidate the information to create datasets containing a higher degree of coherency. 

%\required{Intellectual Merit}
% This should describe the potential of the proposed activity
% to advance knowledge in the field of mathematics. 

\required{Broader Impacts}
% This should describe the potential of the proposed activity
% to benefit society and contribute to the achievement of 
% specific, desired societal outcomes.
% Examples of Broader Imacts include, but are not limited to:
% 1. Full participation of women, persons with disabilities, and 
% underrepresented minorities in science, technology, engineering, 
% and mathematics (STEM); 
% 2. Improved STEM education and educator 
% development at any level; 
% 3. Increased public scientific literacy 
% and public engagement with science and technology; 
% 4. Improved well-being of individuals in society; 
% 5. Development of a diverse, globally competitive STEM workforce;
% 6. Increased partnerships between academia, industry, and others; 
% 7. Improved national security; 
% 8. Increased economic competitiveness of the US; 
% 9. Enhanced infrastructure for research and education.

\begin{tikzpicture}

\timeline{4}
\begin{phases}
    \initialphase{involvement degree=3cm,phase color=blue}
    \phase{between week=1 and 2 in 0.1,
      involvement degree=7cm,phase color=green!50!black}
    \phase{between week=2 and 3 in 0.2,
      involvement degree=6cm,phase color=red!40!black}
    \phase{between week=3 and 4 in 0.5,
      involvement degree=8cm,phase color=red!90!black}
    %\phase{between week=4 and 5 in 0.3,
     % involvement degree=2.5cm,phase color=red!40!yellow}
\end{phases}

\node [xshift=-0.6cm,yshift=1cm,anchor=east, font=\Large\bfseries] at (phase-0.180) {Work Plan};

%Upper labels
\addmilestone{at=phase-0.120,direction=120:1cm, text={Concept}, text options={above}} \addmilestone{at=phase-0.90,direction=90:1.2cm, text={Outline}} \addmilestone{at=phase-1.120,direction=110:1.5cm, text={Hands-On Orientation to SDS}} 
\addmilestone{at=phase-1.70,direction=70:2.5cm, text={Prototype of Parser}}
\addmilestone{at=phase-2.100,direction=100:1cm, text={Creation of data model}} 
\addmilestone{at=phase-3.99,direction=90:2.0cm, text={Consolidation of pipeline}}

%Lower labels
\addmilestone{at=phase-0.270,direction=270:1cm, text={Concept Review}, text options={below}} 
\addmilestone{at=phase-1.270,direction=270:1cm, text={Configuration File Study}} 
\addmilestone{at=phase-2.270,direction=270:0.7cm, text={Interfacing between parser and data store}} 
\addmilestone{at=phase-3.300,direction=270:1.5cm, text={A/B Tests}} 

\end{tikzpicture}
\end{document}
